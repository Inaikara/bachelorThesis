\chapter{绪论}
\section{研究背景和意义}
近年来,随着人工智能的迅速发展,给机器人技能建模与学习领域带来发展新机遇。在15年5月,国务院印发《中国制造2025》明确提出“以加快新一代信息技术与制造业深度融合为主线,以推进智能制造为主攻方向”。到了17年7月,国务院发布了《新一代人工智能发展规划》,将人工智能定位为国家战略,提出了明确的发展目标\cite{Qin2019}。随着中国人口红利的逐渐减少,制造业作为我国的支柱产业,需要尽快转型升级实现智能制造。机器人产业作为智能制造的重要一环,在未来充满广阔的发展空间。

在许多传统大型制造行业,机器人已被广泛应用于生产活动中,例如在许多汽车行业的生产流水线上都实现了高度自动化,取代了枯燥劳累的人力劳动,解放了人类劳动力。过去对机器人的编程方式以手工编程为主,只能覆盖到对于单一场景下的运动需求。机器人示教技术的发展让机器人越来越具备复杂运动的能力,适应了制造业定制化、柔性化的需求,面对一些复杂的、精细化的劳动,机器人也开始逐渐取代人类。机器人智能化程度的提高,也让机器人所面向的客户逐渐从企业转为消费者,机器人不但能为提高生产力效力,还能融入人们的日常生活中。消费者业务的扩展让机器人能走进千家万户,不断拓宽在各个领域的可能性。

人工智能的发展把图像处理技术带到一个新的时代,计算机视觉如今成为炙手可热的课题。短短几年,人脸识别,目标检测,图像分割等图像处理方面新技术应用到了各行各业。由于人类绝大部分对外界的感知信息都是通过眼睛获取的,为了提高机器人的智能化程度和感知能力,计算机视觉和机器人的结合也成为热门的研究方向。

基于视觉信息和动态运动原语的机器人书法系统的研究契合了当下时代发展背景。在科教文娱产业,机器人的应用还处于起步阶段,随着人力成本的提高,在该领域人们对机器人的需求会越来越高。机器人书法系统在应用上能填补这部分空缺。凭借人工智能在计算机视觉上的应用,基于视觉信息的示教方法也有很大的发展前景。

\section{国内外研究现状}
\subsection{机器人示教研究现状}
尽管如今机器人在社会上的应用越来越广泛,在技术上也取得了许多重大的突破,但对于机器人大部分运动任务的编程方式仍然以手工编程为主。机器人示教方法的研究从20世纪80年代开始发展至今,相关研究成果已经相当成熟,其目的就是要实现对机器人运动的自动编程。根据示教方法来进行分类,机器人示教主要分为三种:基于物理交互的示教方法,基于遥操作的示教方法和基于视觉的示教方法。下面将分别介绍三种示教方法的现状与优劣。

1)基于物理交互的示教方法

对于这种示教方法,示教者通常会与机器人进行直接接触,通过拖动机械臂末端等方式进行示教编程。早在1984年,Hogan基于阻抗控制的方法设计出的机械臂动力学模型,就能在无力矩传感器的情形下进行物理交互示教\cite{Hogan1984a}。而根据Goto等人的工作,通过对重力和摩擦力进行补偿,对于高齿轮比的工业机器人示教者也能轻松的实现牵引拖动\cite{Goto2002}。这种示教方法可以让示教者和机械臂有直接的力交互,示教过程相当直观。不过,由于这种示教方法需要示教者和机器人进行近距离的接触,在操作上会受到限制。对于大动力机械臂的示教,近距离的接触也存在一定的安全隐患\cite{Li2021}。在某些恶劣的、高风险的应用场景下,这种示教方法实现起来也相当困难。

2)基于遥操作的示教方法

基于遥操作的示教方法与上述方法的差异是,示教者通常不会和机器人有直接的接触,而是借助遥操作装置实现对机器人的控制。遥操作设备的形式多种多样,Fang等人通过佩戴可穿戴设备将人类手臂的动作映射到七自由度的Baxter机器人上\cite{Fang2019a},Lee等人利用外骨骼实现对双臂机器人的控制\cite{Lee2014},Monferrer等人将虚拟现实技术和机器人的遥操作结合起来,实现对水下移动机器人的控制,兼具沉浸感和可操作性\cite{Monferrer2002}。这种示教方法弥补了基于物理交互示教方法的缺点,可以实现在恶劣的环境下对机器人进行示教学习,例如水下或高温环境。但这种方法存在延时和震颤问题\cite{Yang2017},导致跟踪误差的存在。由于遥操作设备本身较为昂贵并且操作复杂,这种方法的使用门槛也比较高\cite{Li2022}。

3)基于视觉的示教方法

基于视觉的示教方法通过相机、定位光标等方式学习技能,主要依赖上述设备采集到的视觉信息来捕获示教者的运动信息,将技能传递给机器人。基于视觉的示教能实现多种方式的示教。例如我们可以通过Kinect相机识别示教者的深度视觉信息并映射到机械臂上\cite{Xu2018b},也可以脱离示教者本身通过识别图片的轨迹信息实现示教\cite{Li2021},甚至我们可以通过无标签的视频演示的就能让机器人进行模仿学习\cite{Sermanet2018}。这种方法在成本上相对低廉,对于示教者本身的要求也不高,操作较为简便,很容易能获取到大量的样本数据。但这种方法让示教者缺乏对力度信息的反馈,在示教效果上有所欠缺。

\subsection{动态运动原语研究现状}
Giszter等人在1993年对青蛙神经的研究中就开始使用动态系统来描述复杂的运动\cite{Giszter1993b}。Ijspeert和Stefan Schaal实验室在2002年受到在电机控制领域广泛应用的弹簧阻尼模型的启发,将动态系统方程和机器人领域相结合,设计出动态运动原语\cite{Ijspeert2002}。他们在2013年又做了进一步的改进,结合强化学习并论证了改进DMP的方法,让DMP在机器人控制领域得到广泛应用\cite{Ijspeert2013}。

早期传统的DMP只使用一条运动轨迹来对动态系统进行学习,通过构建运动基元库实现复杂运动。Mülling等人通过组合多个DMP系统实现了打乒乓球的动作\cite{Mulling2013}。Paraschos等人利用概率模型搭建起一个统一框架,设计了将多个MP系统组合到一起形成复杂运动的方法\cite{Paraschos2017}\cite{Paraschos2018}。

\subsection{汉字笔画提取研究现状}
对于汉字笔画提取的研究,国内外有许多学者做了不少工作。Liu等人通过B样条曲线拟合实现对汉字笔画的提取\cite{Liu2007a}。Wang等人通过Canny算子和CDT将汉字分解为一组三角形网格,对汉字进行建模,并且使用PBOD分解汉字的轮廓\cite{Wang2013}。Tang等人设计的图像处理算法,过滤噪声和阴影等因素,通过视频实现对汉字笔画的提取\cite{Tang2005}。Zhou等人首先通过八邻域窗口检测模糊区域,然后移除模糊区域将笔画区分开\cite{Zhou2017}。Wu等人结合生成对抗网络和深度强化学习,让机器人能直接从图片学习到书法技能\cite{Wu2020}。上述的汉字笔画提取算法都存在计算量较大的问题,在实时性要求比较高的场合适用性不高。

\section{研究内容和安排}
机器人书法系统主要分为两大部分内容,汉字笔画提取和机器人的技能传递。汉字笔画提取又包含手眼标定,边缘检测,图像增强等内容。机器人技能传递主要实现DMP模型设计。最终将两部分内容结合起来实现机器人书法系统。

\section{论文框架结构}
本文的结构安排如下:

第一章为绪论,主要介绍课题背景和国内外研究现状。

第二章为汉字笔画提取,主要介绍汉字笔画提取的基本方法。

第三章为机器人技能传递,主要介绍DMP模型的应用和实验。

第四章为机器人书法系统设计,主要介绍系统的综合设计。

第五章为机器人书法系统实验,主要介绍实验内容和结果。

第六章为总结和展望,主要介绍对未来研究方向的看法。


