\chapter{绪论}
\section{研究背景和意义}





\section{国内外研究现状}
\subsection{机器人示教方法研究现状}
尽管如今机器人在社会上的应用越来越广泛,在技术上也取得了许多重大的突破,但对于机器人大部分运动任务的编程方式仍然以手工编程为主。机器人示教方法的研究从20世纪80年代开始发展至今,相关研究成果已经相当成熟,其目的就是要实现对机器人运动的自动编程\cite{Chen2020}。根据示教方法来进行分类,机器人示教主要分为三种:基于物理交互的示教方法,基于遥操作的示教方法和基于视觉的示教方法\cite{Zeng2019}。下面将分别介绍三种示教方法的现状与优劣。

1)基于物理交互的示教方法

对于这种示教方法,示教者通常会与机器人进行直接接触,通过拖动机械臂末端等方式进行示教编程。早在1984年,Hogan基于阻抗控制的方法设计出的机械臂动力学模型,就能在无力矩传感器的情形下进行物理交互示教\cite{Hogan1984a}。而根据Goto等人的工作,通过对重力和摩擦力进行补偿,对于高齿轮比的工业机器人示教者也能轻松的实现牵引拖动\cite{Goto2002}。这种示教方法可以让示教者和机械臂有直接的力交互,示教过程相当直观。不过,由于这种示教方法需要示教者和机器人进行近距离的接触,在操作上会受到限制\cite{Zeng2019}。对于大动力机械臂的示教,近距离的接触也存在一定的安全隐患\cite{Li2021}。在某些恶劣的、高风险的应用场景下,这种示教方法实现起来也相当困难。

2)基于遥操作的示教方法

基于遥操作的示教方法与上述方法的差异是,示教者通常不会和机器人有直接的接触,而是借助遥操作装置实现对机器人的控制。遥操作设备的形式多种多样,Fang等人通过佩戴可穿戴设备将人类手臂的动作映射到七自由度的Baxter机器人上\cite{Fang2019a},Lee等人利用外骨骼实现对双臂机器人的控制\cite{Lee2014},Monferrer等人将虚拟现实技术(Virtual Reality, VR)和机器人的遥操作结合起来,实现对水下移动机器人的控制,兼具沉浸感和可操作性\cite{Monferrer2002}。这种示教方法弥补了基于物理交互示教方法的缺点,可以实现在恶劣的环境下对机器人进行示教学习,例如水下或高温环境。但这种方法存在延时和震颤问题\cite{Yang2017},导致跟踪误差的存在。由于遥操作设备本身较为昂贵并且操作复杂,这种方法的使用门槛也比较高\cite{Li2022}。

3)基于视觉的示教方法

基于视觉的示教方法通过相机、定位光标等方式学习技能,主要依赖上述设备采集到的视觉信息来捕获示教者的运动信息,将技能传递给机器人。基于视觉的示教能实现多种方式的示教。例如我们可以通过Kinect相机识别示教者的深度视觉信息并映射到机械臂上\cite{Xu2018b},也可以脱离示教者本身通过识别图片的轨迹信息实现示教\cite{Li2021},甚至我们可以通过无标签的视频演示的就能让机器人进行模仿学习\cite{Sermanet2018}。这种方法在成本上相对低廉,对于示教者本身的要求也不高,操作较为简便,很容易能获取到大量的样本数据。但这种方法让示教者缺乏对力度信息的反馈,在示教效果上有所欠缺。

\subsection{动态运动原语研究现状}
Giszter等人在1993年对青蛙神经的研究中就开始使用动态系统来描述复杂的运动\cite{Giszter1993b}。Ijspeert和Stefan Schaal实验室在2002年受到在电机控制领域广泛应用的弹簧阻尼模型的启发,将动态系统方程和机器人领域相结合,设计出动态运动原语\cite{Ijspeert2002}。他们在2013年又做了进一步的改进,结合强化学习并论证了改进DMP的方法,让DMP在机器人控制领域得到广泛应用\cite{Ijspeert2013}。

早期传统的DMP只使用一条运动轨迹来对动态系统进行学习,通过构建运动基元库实现复杂运动。Mülling等人实现通过组合多个DMP系统实现了打乒乓球的动作\cite{Mulling2013}。





\subsection{汉字笔画提取研究现状}

\section{研究内容和安排}


