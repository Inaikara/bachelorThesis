\chapter{绪论}
\section{研究背景和意义}





\section{国内外研究现状}
\subsection{机器人示教方法研究现状}
尽管如今机器人在社会上的应用越来越广泛,在技术上也取得了许多重大的突破,但对于机器人大部分运动任务的编程方式仍然以手工编程为主\cite{Song2021}。机器人示教方法的研究从20世纪80年代开始发展至今,相关研究成果已经相当成熟,其目的就是要实现对机器人运动的自动编程\cite{Chen2020}。根据示教方法来进行分类,机器人示教主要分为三种:基于物理交互的示教方法,基于遥操作的示教方法和基于视觉的示教方法\cite{Zeng2019}。下面将分别介绍三种示教方法的现状与优劣。

1)基于物理交互的示教方法

对于这种示教方法,示教者通常会与机器人进行直接接触,通过拖动机械臂末端等方式进行示教编程。早在1984年,Hogan基于阻抗控制的方法设计出的机械臂动力学模型,就能在无力矩传感器的情形下进行物理交互示教\cite{Hogan1984a}。而根据Goto等人的工作,通过对重力和摩擦力进行补偿,对于高齿轮比的工业机器人示教者也能轻松的实现牵引拖动\cite{Goto2002}。这种示教方法可以让示教者和机械臂有直接的力交互,示教过程相当直观\cite{Zeng2019}。不过,由于这种示教方法需要示教者和机器人进行近距离的接触,在操作上会受到限制。对于大动力机械臂的示教,近距离的接触也存在一定的安全隐患。在某些恶劣的、高风险的应用场景下,这种示教方法实现起来也相当困难。

2)基于遥操作的示教方法



\subsection{动态运动原语研究现状}

\subsection{汉字笔画提取研究现状}

\section{研究内容和安排}


