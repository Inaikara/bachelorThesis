%\documentclass[unicode,master]{scutthesis} % 草稿封面,硕士则添加选项master,博士则去掉。使用正式封面时注释该行
\documentclass[unicode,master,pdfcover]{scutthesis} %   % 论文正式封面,pdfcover为可选项,终稿再添加,使用草稿封面时注释该行
\usepackage{fontspec,color,array,longtable,graphicx} 
\usepackage{anyfontsize} %消除字体警告
\usepackage{enumitem}
%%%%%%%%%%%%%%%%%%%%%%%%%%%%%%%%%%%%%%%%%%%%%%%%%%%%%%%%%%%%%%%%%%%%%%%%%%%%%%%%%%——by MCH
%编译范围
% \includeonly{chapter04}
%参考文献设置
\usepackage[backend=biber,style=gb7714-2015,gbalign=gb7714-2015,gbpub=false,gbnamefmt = lowercase]{biblatex}
\addbibresource[location=local]{MyLibrary.bib} % 如果在其他盘,改为相对路径。比如F盘,改为:F/MyLibrary.bib
\addbibresource[location=local]{mybibfile2.bib} % 无论什么来源的bib文件,只要由参考文献的BibTeX组成,都可以使用此模板。参考文献的BibTeX获取方法可百度
%页眉页脚设置
\usepackage{fancyhdr}
\usepackage{listings}
\usepackage{xunicode}
\renewcommand{\lstlistingname}{列表}
\pagestyle{fancy}
\fancyfoot[C]{\headfont\thepage}
\renewcommand{\chaptermark}[1]{\markboth{\chaptername\ #1}{}}
\renewcommand{\sectionmark}[1]{\markright{\thesection\ #1}}
\fancyhead[RE]{}
\fancyhead[RO]{}
\fancyhead[LE]{}
\fancyhead[LO]{}
\fancyhead[CO]{\headfont{\leftmark}}
\fancyhead[CE]{\headfont{华南理工大学学士学位论文}}% 
\renewcommand{\headrulewidth}{1.5pt}
\renewcommand{\footrulewidth}{0pt}
%%%%%%%%%%%%%%%%%%%%%%%%%%%%%%%%%%%%%%%%%%%%%%%%%%%%%%%%%%%%%%%%%%%%%%%%%%%%%%%%%%
\usepackage[unicode=true,bookmarks=true,bookmarksnumbered=true,bookmarksopen=false,breaklinks=false,pdfborder={0 0 1},backref=false,colorlinks=true]{hyperref}
\hypersetup{%%pdftitle={LaTeX模板使用说明},
%%	pdfauthor={蒙超恒},
%%	pdfsubject={华南理工大学硕士学位论文},
%%	pdfsubject={华南理工大学博士学位论文},
%%	pdfkeywords={PDF关键字1;PDF关键字2},
%%		linkcolor=black, anchorcolor=black, citecolor=black, filecolor=black, menucolor=black, urlcolor=black, pdfstartview=FitH}% 黑白,提交版
	linkcolor=blue, anchorcolor=black, citecolor=red, filecolor=magenta, menucolor=red, urlcolor=magenta, pdfstartview=FitH}% 彩色

\makeatletter
%%%%%%%%%%%%%%%%%%%%%%%%%%%%%% LyX specific LaTeX commands.
\providecommand{\LyX}{\texorpdfstring%
	{L\kern-.1667em\lower.25em\hbox{Y}\kern-.125emX\@}
	{LyX}}
%% Because html converters don't know tabularnewline
\providecommand{\tabularnewline}{\\}
\makeatother
\begin{document}
	%%%%%%%%%%%%%草稿封面设置%%%%%%%%%%%%%使用“正式封面”时不需要理会这部分
	\title{动态运动原语}	
	\author{许文晋}	
	\supervisor{指导教师:杨辰光\ 教授}	
	\institute{华南理工大学}	
	\date{2023年5月20日}
	%%%%%%%%%%%%%%%%%%%%%%%%%%%%%%%%%%%%%
	%%\maketitle	
	\frontmatter	%此后为罗马数字页码,页面类型为plain
	\chapter{摘\texorpdfstring{\quad}{}要}


\keywordsCN{\LaTeX{};论文}

\chapter{Abstract}
	

\keywordsEN{\LaTeX{}; Paper} % 中英文摘要
	%%%%%%%%%%%%%%%%%%%%%%%%%%%%%%%%%%%%%%%%%%%%%%%%
	% 目录、表格目录、插图目录这几个字本身的大纲级别是一级的,即和章名有相同的字号字体。目录表的内容通过titletoc宏包在。cls文件设置了。
	%\cleardoublepage % pdfbookmark可能需要这一条才能正常工作
	\pdfbookmark{\contentsname}{toc} %为目录添加pdf文件书签
	\tableofcontents	%目录
	% \listoffigures	%插图目录(可选)
	% \listoftables	%表格目录(可选)
	

	\begingroup
		\renewcommand*{\addvspace}[1]{}
		\newcommand{\loflabel}{图} 
		\renewcommand{\numberline}[1]{\loflabel~#1\hspace*{1em}}	
		%\listoffigures
		
		\newcommand{\lotlabel}{表}
		\renewcommand{\numberline}[1]{\lotlabel~#1\hspace*{1em}}
		%\listoftables
	\endgroup


	%%%%%%%%%%%%%%%%%%%%%%%%%%%%%%%%%%%%%%%%%%%%%%%%%
	%\include{symbols}	% 符号对照表(可选)
	%\chapter{英文缩略词}
【本节论文规范为可选,如果你的论文没有相关内容那么去除这一节;如果有,则删除这一行注释。】
\begin{table}
	\centering{}%
	\begin{tabular}{ccc}
		SCUT  & South China University of Technology & 华南理工大学\tabularnewline
		&  & \tabularnewline
		&  & \tabularnewline
		&  & \tabularnewline
		&  & \tabularnewline
	\end{tabular}
\end{table} 	% 缩略词	
	
	\mainmatter %此后为阿拉伯数字页码
	
    %%%%%%%%%%%%%%%%%%%%%%%%%%%%%%%%%%%%%%%%%%%%%%页眉页脚设置 ——by MCH 
    \fancypagestyle{plain}{
    	\pagestyle{fancy}
    }	% 每章的第一页会默认使用plain,没有页眉。通过重定义plain为fancy解决
    \pagestyle{fancy}	%设置页眉页脚为fancy
    %%%%%%%%%%%%%%%%%%%%%%%%%%%%%%%%%%%%%%%%%%%%%%分章节,结合导言区的\includeonly命令可仅编译部分章节,编译时不用切换界面,直接在相应章节编译即可。
	\chapter{绪论}
\section{研究背景和意义}
哈哈hahahahahaha
\section{国内外研究现状}

\section{研究内容和安排}


%第一章
	%\chapter{第二章}%第二章
	%\chapter{第三章}%第三章
	%\chapter{第四章}%第四章
	% 自行根据需要添加章节。

	\backmatter %章节不编号但页码继续
	%%%%%%%%%%%%%%%%%%%%%%%%%%%%%%%%%%%%%%%%%%%%%%%%%%%%%%%%%%%%%%    微调,使得后续章节的页眉不带章号——by MCH
	\renewcommand{\chaptermark}[1]{\markboth{#1}{}}
	%%%%%%%%%%%%%%%%%%%%%%%%%%%%%%%%%%%%%%%%%%%%%%%%%%%%%%%%%%%%%%
	%\chapter{结\texorpdfstring{\quad}{}论}
 %结论
	 %%%%%%%%%%%%%%%%%%%%%%%%%%%%%%%%%%%%%%%%%%%%%% bibtex参考文献设置  (原版)
%%	\bibliographystyle{scutthesis}
%%	\bibliography{F:/MyLibrary}
	%%%%%%%%%%%%%%%%%%%%%%%%%%%%%%%%%%%%%%%%%%%%%%
	%%%%%%%%%%%%%%%%%%%%%%%%%%%%%%%%%%%%%%%%%%%%%% biber参考文献设置	——by MCH
	%\renewcommand*{\bibfont}{\refbodyfont}			% 设置文献著录字号比正文小一号(五号),需要小四号请注释该行. % 不推荐使用small,而是使用cls文件中精确定义了的字号。
	\phantomsection % “目录”中的链接能正确跳转,需要添加 \phantomsection 否则点击参考文献会跳转到结论
	\addcontentsline{toc}{chapter}{参考文献}	%目录中添加参考文献
	\printbibliography	% 参考文献著录
 	%%%%%%%%%%%%%%%%%%%%%%%%%%%%%%%%%%%%%%%%%%%%%%
 	% 只有一个附录
 	%%%%%%%%%%%%%%%%%%%%此部分为附录环境代码,是比较笨的方法来适应论文撰写规范%%%%%%%%%%%%%%%%%%%%%%%%%%%%%%%%%%%%%%
%对只有一个附录,标题不编号比较美观。
%%%%%%%%%%%%%%%%%%%%%%%%%%%%%%%%%%%%%%%%%%%%%%%%%%%%%%%%%%%%%%%%%%%%%%%%%%%%%%%%%%%%%%%%%%%%%%%%%%%%%%%%%%%%
\setcounter{chapter}{1} %从1开始编号
\setcounter{section}{0}
\setcounter{equation}{0}
\setcounter{table}{0}   
\setcounter{figure}{0}
\chapter{附\texorpdfstring{\quad}{}录} %附录
%%%%%%%%%%%%%%%%%%%%%%%%%%%%%%%%%%%%%%%%%%%%%%%%%%%%%%%%%%%%%%%%%%%%%%%%%%%%%%%%%%%%%%%%%%%%%%%%%%%%%%%%
%%%%%%%%%%%%以下为用户代码,用于撰写您的论文%%%%%%%%%%%%%%%%%%%%%%%%%%%%%%%%%%%%%%%%%%%%%%%%%%%%%%%%%%%%%%



 	% 有多个附录
	%\include{appendix1} %附录1
	%\include{appendix2} %附录2
 	%%%%%%%%%%%%%%%%%%%
	%\include{pub} %成果
	%\chapter{致\texorpdfstring{\quad}{}谢}
 %致谢
\end{document}
